\usepackage[a4paper,left=2.8cm,right=2.8cm,top=2.8cm,bottom=2.8cm]{geometry}
%\documentclass[paper=a4, fontsize=11pt]{scrartcl} % A4 paper and 11pt font size

\usepackage[T1]{fontenc} % Use 8-bit encoding that has 256 glyphs
%\usepackage{fourier} % Use the Adobe Utopia font for the document - comment this line to return to the LaTeX default
\usepackage[english]{babel} % English language/hyphenation
\usepackage[utf8]{inputenc}  %allows non-English characters
\usepackage{amsmath,amsfonts,amsthm, amssymb} % Math packages
\usepackage{float}


\usepackage{indentfirst}
\usepackage{caption}
\usepackage{subcaption}
\usepackage{tocbibind}

\usepackage{verbatim}
\usepackage{titlesec}
\usepackage{longtable}
\usepackage{listings}
\usepackage{fancyvrb}
\fvset{frame=single,framesep=1mm,commandchars=\\\{\}}
\usepackage[pdftex]{hyperref,color,graphicx}
%\usepackage[pdftex]{hyperref}
\usepackage{import}
%\usepackage[pdftex,bookmarks,colorlinks]{hyperref}
\usepackage{url}
\usepackage[usenames,dvipsnames]{xcolor}
\usepackage{rotating}
\usepackage{colortbl}
\usepackage{array}
\usepackage[Conny]{fncychap}
\usepackage{listings}
\usepackage{caption}
\renewcommand{\lstlistingname}{Code}
\makeatletter
\renewcommand{\DOCH}{%
    \vskip -4.5\baselineskip
    \mghrulefill{3\RW}\par\nobreak
    %\vskip -0.5\baselineskip
    \mghrulefill{\RW}\par\nobreak
    \CNV\FmN{\@chapapp}\space \CNoV\thechapter
    \par\nobreak
    \vskip -0.5\baselineskip
   }
  \renewcommand{\DOTI}[1]{%
    \mghrulefill{\RW}\par\nobreak
    \CTV\FmTi{#1}\par\nobreak
    \vskip 30\p@
    }
  \renewcommand{\DOTIS}[1]{%
    \mghrulefill{\RW}\par\nobreak
    \CTV\FmTi{#1}\par\nobreak
    \vskip 60\p@
    }
\makeatother

% subfigure and wrapping on figures
\usepackage{subcaption}
\usepackage{wrapfig}

\usepackage[intoc]{nomencl}
\makenomenclature

\hypersetup{
pdftitle={Anonymous Threshold Signatures},
pdfauthor={Hlad Colic, Petar},
%pdfsubject={},
%pdfkeywords={},
colorlinks,
linkcolor=black,
citecolor=black,
urlcolor=blue,
unicode=false
}

\usepackage{fancyhdr} % Custom headers and footers
\pagestyle{fancyplain} % Makes all pages in the document conform to the custom headers and footers
\renewcommand{\sectionmark}[1]{\markright{\thesection\ #1}}
\fancyfoot[LE,RO]{\thepage}
\fancyfoot[C]{}
\fancyhead[RO]{\footnotesize\bfseries}
\fancyhead[LE]{\footnotesize\bfseries}
%\fancyhead[LO,RE]{}

\setcounter{secnumdepth}{4} 
\graphicspath{{../figs/}}

%Include degree symbol in plain text
\usepackage{gensymb}


%Notes on the page (toggle last lines)
\usepackage[colorinlistoftodos]{todonotes}
\setlength{\marginparwidth}{2cm}
\newcommand{\mytodo}[1]{\todo[size=\footnotesize]{#1}}

%\usepackage{biblatex}
\usepackage{enumitem}

\usepackage{flafter}
\usepackage{multirow}
\newcolumntype{M}[1]{>{\arraybackslash}p{#1}}
\newcolumntype{N}{@{}m{0pt}@{}}
\usepackage{pbox}
\usepackage[gen]{eurosym}
%\newcommand{\euro}{euroWARNING}

\lstdefinestyle{tt}{
breakatwhitespace=true,
breaklines     = true,
  frame=top,frame=bottom,
  basicstyle=\small\normalfont\sffamily,    % the size of the fonts that are used for the code
   stepnumber=1,
   numbersep=10pt,                     % how far the line-numbers are from the code
   numbers=left,
   numberstyle=\tiny\color{gray},
  tabsize=2,                              % tab size in blank spaces                     %
  breaklines=true,                        % sets automatic line breaking
  captionpos=t,                           % sets the caption-position to top
    showspaces=false,           % Leerzeichen anzeigen ?
  showtabs=false,             % Tabs anzeigen ?
  xleftmargin=17pt,
  framexleftmargin=17pt,
  framexrightmargin=17pt,
  framexbottommargin=5pt,
  framextopmargin=5pt,
  showstringspaces=false
 }
\DeclareCaptionFormat{listing}{#1#2#3}
\captionsetup[lstlisting]{format=listing,singlelinecheck=false, margin=0pt, font={sf}}

\usepackage{adjustbox}

\usepackage{pdfpages}
\usepackage{afterpage}

%shortcuts for typing variance and expectation
\newcommand{\E}{\mathrm{E}}
\newcommand{\Var}{\mathrm{Var}}
\newcommand{\RR}{{\mathbb R}}
\newcommand{\ZZ}{{\mathbb Z}}
\newcommand{\PP}{{\mathcal P}}
\newcommand{\A}{{\mathcal A}}
\newcommand{\con}{\! \parallel \!}

\theoremstyle{plain}
\newtheorem{thm}{Theorem}[section]
\newtheorem{cor}[thm]{Corollary}
\newtheorem{prop}[thm]{Proposition}

\theoremstyle{definition}
\newtheorem{prob}[thm]{Problem}
\newtheorem{defn}[thm]{Definition} 
\newtheorem{exmp}[thm]{Example}
\newtheorem{remk}[thm]{Remark}
