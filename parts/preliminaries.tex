\chapter{Preliminaries}

Some cryptographic preliminaries and other definitions.

\newpage

\section{Bilinear pairings}
\cite{DiHe76}

Let $G_1$ and $G_2$ be two (multiplicative) cyclic groups of prime order $q$. Let $g_1$ be a fixed generator of $G_1$ and $g_2$ be a fixed generator of $G_2$.

\defn Computation Diffie-Hellman (CDH) Problem: Given a randomly chosen $g \in G_1$, $g^a$, and $g^b$ (for unknown randomly chosen $a,b \in \ZZ_q$), compute $g^{ab}$.

\defn Decision Diffie-Hellman (CDH) Problem: Given randomly chosen $g \in G_1$, $g^a$, $g^b$, and $g^c$ (for unknown randomly chosen $a,b,c \in \ZZ_q$), decide whether $c = ab$. (If so, $(g, g^a, g^b, g^c)$ is called a valid Diffie-Hellman tuple.)

\defn Computational co-Diffie-Hellman (co-CDH) Problem on $(G_1, G_2)$: Given $g_2, g^a_2 \in G_2$ and $h \in G_1$ as input, compute $h^a \in G_1$.

\defn Decision co-Diffie-Hellman (co-DDH) on $(G_1, G_2)$: Given $g_2, g^a_2 \in G_2$ and $h, h^b \in G_1$ as input, decide whether $a=b$. If so, we say that $(g_2, g^a_2, h, h^a)$ is a co-Diffie-Hellman tuple.

\defn Bilinear map: Let $G_T$ be an additional group such that $ |G_1| = |G_2| = |G_T| $. A bilinear map is a map $e: G_1 \times G_2 \rightarrow G_T$ with the following properties:
\begin{enumerate}

\item Bilinear: for all $u \in G_1, v \in G_2$ and $a,b \in \ZZ$, $e(u^a, v^b) = e(u,v)^{ab}$.
\item Non-degenerate: $e(g_1, g_2) \neq 1$.

\end{enumerate}

\newpage

\section{Homomorphic PKE}

\section{Digital Signatures}
To ensure integrity of data in communications and authentication, the concept of digital signatures was developed. 

A digital signature scheme consists of 3 algorithms:
\begin{itemize}
    \item \textbf{Key generation}: on input of a security parameter $k$ (usually the length), outputs a pair $(sk, pk)$ of secret and public keys.
    \item \textbf{Signature}: given an input message $m$ and the secret key $sk$, outputs a signature $\sigma$.
    \item \textbf{Verification}: given an input message $m$, a signature $\sigma$ on the message and a public key $pk$, ouptuts whether the signature is valid or not.
\end{itemize}

A signature scheme must satisfy the following properties:
\begin{itemize}
    \item \textbf{Correctness}: A signature generated with the signing algorithm must always be accepted by the verifier.
    \item \textbf{Unforgeability}: Only a user can sign messages on behalf of himself.
    \item \textbf{Non-repudiation}: 
\end{itemize}

\subsection{Examples}
\subsubsection{ElGamal}
Let $H$ be a collision-resistant hash function. Let $p$ be a large prime such that the \textit{discrete logarithm problem} is difficult over $\ZZ_p$. Let $g$ be a randomly chosen generator of $\ZZ_p^\ast$

\paragraph{Key Generation}

Randomly choose a secret key $x \in \ZZ_p^\ast$, and compute the public key $y = g^x$.

\paragraph{Signature}

To sign a message $m$, the signer chooses a random $k \in \ZZ_p^\ast$.
Compute $ r = g^k$. To compute $s$, the following equation must be satisfied: $g^{H(m)} = g^{xr} g^{ks}$.
So $s = \left( H(m) - xr \right) k^{-1} \quad (\text{mod } p-1)$

If $s=0$, it starts over again with a different $k$.

The pair $(r,s)$ is the digital signature for $m$.

\paragraph{Verification}

Check $g^{H(m)} = y^r r^s$

The use of $H( \cdot )$ prevents an existential forgery attack.

\subsubsection{Boneh-Lynn-Shacham (BLS)}
Let $G,G_T$ be groups of prime order $p$. Let $g$ be a generator of $G$. Let $e:G \times G \rightarrow G_T$ be a non-degenerate bilinear pairing.
\paragraph{Key Generation}

Randomly choose a secret key $x \in \ZZ_p$. The public key will be $y = g^x$.

\paragraph{Signature}

The signature on $m$ is $\sigma = H(m)^x$.

\paragraph{Verification}

Given a signature $\sigma$ and a public key $g^x$, it verifies that $e(\sigma,g) = e \left( H(m), g^x \right)$.

\section{Signature Aggregation}
Explain how different signature schemes allow aggregation of n signatures on n messages from n signers.

\newpage
\section{Group Signatures}
Use of aggregation: group signatures. They are used to sign on behalf of the group, prove group membership.

The easyest way is to give everyone the secret key, so they can sign. But this would let any colluded user to share the secret key to other parties, which is not admissible.

Some group signatures need what is called a Dealer, which will deal with the keys.

Examples of Group signatures:

\section{Shamir secret Sharing}
Shamir secret sharing described in \cite{Sham79}.

The scheme is based on polynomial interpolation.

Let $p$ be a large prime number. All operations are done in $\ZZ_p$

We want to share a secret $s$ into $n$ shares so that the secret can be recovered with any $k$ distinct shares.

Let $q(x) = a_0 + a_1 x + \cdots + a_{k-1} x^{k-1}$ be a random polynomial of degree $k-1$ in which $a_0 = s$.

Each participant $i$ in $\PP = \{ 1, ... , n \}$ is given a different random number $x_i \in \ZZ_p$ which identifies the participant.
Then, each participant $i$ is given the share $y_i = q(x_i)$.

To recover the secret, we only need $k$ different shares. Let $P \subset \PP$ be any subset of $k$ participants. Then

$$
    q(x) = \sum_{i \in P} y_i \prod_{\substack{j \in P \\ i \neq j}} \frac{x-x_j}{x_i-x_j}
$$

Then, $s = q(0)$.

\section{Threshold signature scheme using Shamir}


\section{Anonymity}
The concept of anonimity is used in so many ways

You could use ring signatures which proofs the knowledge of a 1-out-of-N secret key. This could be useful for a small amount of signatures. But it is not useful to provide general anonymity.

A signature scheme provides anonymity if:
\begin{itemize}
    \item Unlinkability: cannot decide whether two different signatures were signed by the same user.
    \item 
\end{itemize}