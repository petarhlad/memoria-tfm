\chapter{Preliminaries}

Some cryptographic preliminaries and other definitions.

\section{Homomorphic Public Key Encryption}
Let ¿? be a public key cryptosystem with encryption algorithm (function?) $\mathcal{E}$ and decryption algorithm (function?) $\mathcal{D}$ where:
$m' \leftarrow \mathcal{E}_{pk} (m)$ and $m \leftarrow \mathcal{D}_{sk} (m')$

Let $(G,\bullet)$ and $(H,\circ)$ be two groups such that $\mathcal{E}_{pk}: G \rightarrow H$.

Then, ¿? is a Homomorphic Public Key Encryption Scheme if
$$ \mathcal{E}_{sk} (m_1 \bullet m_2) \leftarrow \mathcal{E}_{sk} (m_1) \circ \mathcal{E}_{sk} (m_2)$$

Example:
\textbf{ElGamal}


\section{Oblivious Polynomial Evaluation}
Oblivious polynomial evaluation is a protocol involving two parties, a sender whose input is a polynomial $P$, and a receiver whose input is a value $\alpha$. At the end of the protocol, the receiver learns $P(\alpha)$ and the sender learns nothing.

\subsection{Noisy Polynomial problem}

\citeauthor{OPE} proposed in \cite{OPE} a protocol for Oblivious Polynomial Evaluation that relies on the hardness of the Noisy Polynomial Reconstruction problem.

The description of the protocol follows:

\begin{itemize}
    \item Input:
    
    \begin{itemize}
        \item Sender: a polynomial $P(y) = \sum_{i=0}^{d_P} b_i y^i$ of degree $d_P$ in the field $\FF$.
        \item Receiver: a value $\alpha \in \FF$
    \end{itemize}
    
    \item Output:
    
    \begin{itemize}
        \item Sender: nothing.
        \item Receiver: $P(\alpha)$
    \end{itemize}
    
    \item Protocol security parameters: $m$, $k$.
    
\end{itemize}

Protocol:

1. The sender hides $P$ in a bivariate polynomial:

The sender generates a random masking polynomial $P_x (x)$ of degree $d$ s.t. $P_x (0) = 0$.
$$P_x (x) = \sum_{i=1}^{d} a_i x^i$$
where $d = k \cdot d_P$.

The sender defines a bivariate polynomial
$$ Q(x,y) = P_x(x) + P(y) = \sum_{i=1}^{d} a_i x^i + \sum_{i=0}^{d_P} b_i y^i $$

2. The receiver hides $\alpha$ in a univariate polynomial:

The receiver chooses a random polynomial $S$ of degree $k$, such that $S(0) = \alpha$.

Define $R(x) = Q( x, S(x))$.


\subsection{Homorphic PKE}

\section{Secret Sharing}
Definition of access structures and secret sharing

\cite{Phillips1992}
\cite{BlSt97}

An access structure $\Gamma$ is the set of all subsets of $\PP$ that can recover the secret.

\defn Let $\PP := \{ P_1, \dots, P_n \}$ be a set of participants. A \textit{monotone access structure} $\Gamma$ on $\PP$ is a subset $\Gamma \subseteq 2^{\PP}$, which is monotone increasing $$ A \in \Gamma, \quad A \subseteq A' \subseteq \PP \Rightarrow A' \in \Gamma $$

\defn Let $\PP := \{ P_1, \dots, P_n \}$ be a set of participants and let $A \subseteq 2^{\PP}$. The \textit{closure} of $A$, denoted $\cl (A)$, is the set $$ \cl (A) = \{ C: \exists B \in A \text{ s.t. } B \subseteq C \subseteq \PP \}$$

For a monotone access structure $\Gamma$ we have $\Gamma = \cl (\Gamma)$.

\defn Let $\Gamma$ be an access structure on a set of participants $\PP$. $B \in \Gamma$ is a \textit{minimal} qualified set if $A \notin \Gamma$ whenever $A \subsetneq B$.

\defn Let $\Gamma$ be an access structure on a set of participants $\PP$. The family of minimal qualified sets $\Gamma_{0}$ of $\Gamma$ is called the \textit{basis} of $\Gamma$.

For a basis $\Gamma_0$ of an access structure $\Gamma$ we have $\Gamma = \cl (\Gamma_{0})$

\defn An access structure $\Gamma$ is \textit{trivial} if either $\Gamma = 2^{\PP}$ or $\Gamma = \{ \PP \}$.

Let $\mathcal{K}$ be a set of $q$ elements called \textit{secret keys}, and let $\mathcal{S}$ be a finite set whose elements are called \textit{shares}. Let $D$ be a \textit{dealer} who wants to share a secret key $\mathbf{k} \in \mathcal{K}$ among the participants in $\PP$. 

\defn A \textit{distribution rule} is a function $f : \PP \cup \{ D \} \rightarrow \KK \cup \SS$ which satisfies the conditions $f(D) \in \KK$ and $f(P_i) \in \SS$ for $i = 1,2, \dots, n$.

Secret sharing schemes will be represented by a collection of distribution rules, which represent a possible distribution of shares to the participants where $f(D)$ is the secret key being shared and $f(P_i)$ is the share given to $P_i$.

\defn Let $\mathscr{F}$ be a family of distribution rules, and let $\mathbf{k} \in \KK$. Then $\mathscr{F}_{\mathbf{k}} := \{ f \in \mathscr{F} : f(F) = \mathbf{k}\}$ is the family of all distribution rules having $\mathbf{k}$ as secret.

If $\kk \in \KK$ is the secret that $D$ wants to share, then $D$ will chose a distribution rule $f \in \F_{\KK}$ uniformly at random.


Let $\{p_{\KK} (\kk) \}_{\kk \in \KK}$ be a probability distribution on $\KK$, and let a collection of distribution rules for secrets in $\KK$ be fixed.

\defn A \textit{perfect secret sharing scheme}, with respect to a monotone access structure $\Gamma \subseteq 2^{\PP}$, is a collection of distribution rules that satisfy the following two properties:
\begin{enumerate}
    \item If a subset $A \in \Gamma$ of participants pool their shares, then they can determine the value of the secret $\kk$.
    \item If a subset $A \notin \Gamma$ of participants pool their shares, then they can determine nothing about the value of the secret $\kk$. Formally, if $A \notin \acc$ then for all $a = \{ (P_i, s_i): P_i \in A \text{ and } s_i \in \SS \}$ with $p(a)>0$, and for all $\kk \in \KK$, it holds $p(\kk \vert a) = p_{\KK}(\kk)$. In other words, the \textit{a priori} probability of the value of $\kk$ does not change after knowing the shares held by $A$.
\end{enumerate}

\defn An \textit{ideal secret sharing scheme} is a secret sharing scheme for which $\vert \KK \vert = \vert \SS \vert$. An access structure admitting an ideal secret sharing scheme will be referred as \textit{ideal access structure}.

\thm \sout{Let $\acc$ be an access structure on a set of participants $\PP$. An ideal anonymous secret sharing scheme for $\acc$ exists if and only if either $\acc$ is a $(1,\vert \PP \vert)$ threshold structure, a $(\vert \PP \vert, \vert \PP \vert)$ threshold structure, or the closure of a complete bipartite graph.}

Shamir as example

Anonymous secret sharing


\section{Digital Signatures}
Definition and examples

\section{Group Signatures}
(With linkability)

\section{Threshold Digital Signatres}
Definition and example (shamir)


\newpage




\newpage

\section{Bilinear pairings}
\cite{DiHe76}

Let $G_1$ and $G_2$ be two (multiplicative) cyclic groups of prime order $q$. Let $g_1$ be a fixed generator of $G_1$ and $g_2$ be a fixed generator of $G_2$.

\defn Computation Diffie-Hellman (CDH) Problem: Given a randomly chosen $g \in G_1$, $g^a$, and $g^b$ (for unknown randomly chosen $a,b \in \ZZ_q$), compute $g^{ab}$.

\defn Decision Diffie-Hellman (DDH) Problem: Given randomly chosen $g \in G_1$, $g^a$, $g^b$, and $g^c$ (for unknown randomly chosen $a,b,c \in \ZZ_q$), decide whether $c = ab$. (If so, $(g, g^a, g^b, g^c)$ is called a valid Diffie-Hellman tuple.)

\defn Computational co-Diffie-Hellman (co-CDH) Problem on $(G_1, G_2)$: Given $g_2, g^a_2 \in G_2$ and $h \in G_1$ as input, compute $h^a \in G_1$.

\defn Decision co-Diffie-Hellman (co-DDH) on $(G_1, G_2)$: Given $g_2, g^a_2 \in G_2$ and $h, h^b \in G_1$ as input, decide whether $a=b$. If so, we say that $(g_2, g^a_2, h, h^a)$ is a co-Diffie-Hellman tuple.

\defn Bilinear map: Let $G_T$ be an additional group such that $ |G_1| = |G_2| = |G_T| $. A bilinear map is a map $e: G_1 \times G_2 \rightarrow G_T$ with the following properties:
\begin{enumerate}

\item Bilinear: for all $u \in G_1, v \in G_2$ and $a,b \in \ZZ$, $e(u^a, v^b) = e(u,v)^{ab}$.
\item Non-degenerate: $e(g_1, g_2) \neq 1$.

\end{enumerate}

\defn A Gap co-Diffie-Hellman (co-GDH) group pair is a pair of groups $(G_1, G_2)$ on which co-DDH is easy but co-CDH is hard. When $(G_1, G_1)$ is a co-GDH group pair, we say $G_1$ is a Gap group (GDH).

\remk If there s a bilinear map on $G_1, G_2$, then they are a co-GDH group pair. If there is a bilinear map over $G_1 \times G_1$, then $G_1$ is a gap group, since one can use the bilinear map to solve the DDH problem.

\section{Oblivious Polynomial Evaluation}

Oblivious polynomial evaluation is a protocol involving two parties, a sender whose input is a polynomial $P$, and a receiver whose input is a value $\alpha$. At the end of the protocol, the receiver learns $P(\alpha)$ and the sender learns nothing.

Generic Protocol:

\begin{itemize}
    \item Input:
    
    \begin{itemize}
        \item Sender: a polynomial $P(y) = \sum_{i=0}^{d_P} b_i y^i$ of degree $d_P$ in the field $\FF$.
        \item Receiver: a value $\alpha \in \FF$
    \end{itemize}
    
    \item Output:
    
    \begin{itemize}
        \item Sender: nothing.
        \item Receiver: $P(\alpha)$
    \end{itemize}
    
    \item Protocol security parameters: $m$, $k$.
    
\end{itemize}

Protocol:

1. The sender hides $P$ in a bivariate polynomial:

The sender generates a random masking polynomial $P_x (x)$ of degree $d$ s.t. $P_x (0) = 0$.
$$P_x (x) = \sum_{i=1}^{d} a_i x^i$$
where $d = k \cdot d_P$.

The sender defines a bivariate polynomial
$$ Q(x,y) = P_x(x) + P(y) = \sum_{i=1}^{d} a_i x^i + \sum_{i=0}^{d_P} b_i y^i $$

2. The receiver hides $\alpha$ in a univariate polynomial:

The receiver chooses a random polynomial $S$ of degree $k$, such that $S(0) = \alpha$.

Define $R(x) = Q( x, S(x))$.

\cite{OPE}

\section{Homomorphic PKE}

\section{Digital Signatures}
To ensure integrity of data in communications and authentication, the concept of digital signatures was developed. 

A digital signature scheme consists of 3 algorithms:
\begin{itemize}
    \item \textbf{Key generation}: on input of a security parameter $k$ (usually the desired length for the keys), outputs a pair $(sk, pk)$ of secret and public keys.
    \item \textbf{Signature}: given an input message $m$ and the secret key $sk$, outputs a signature $\sigma$.
    \item \textbf{Verification}: given an input message $m$, a signature $\sigma$ on the message and a public key $pk$, outputs whether the signature is valid or not.
\end{itemize}

A signature scheme must satisfy the following properties:
\begin{itemize}
    \item \textbf{Correctness}: A signature generated with the signing algorithm must always be accepted by the verifier.
    \item \textbf{Unforgeability}: Only a user can sign messages on behalf of himself.
    \item \textbf{Non-repudiation}: 
\end{itemize}

\subsection{Examples}
\subsubsection*{ElGamal}
\cite{elGamal85}
Let $H$ be a collision-resistant hash function. Let $p$ be a large prime such that the \textit{discrete logarithm problem} is difficult over $\ZZ_p$. Let $g$ be a randomly chosen generator of $\ZZ_p^\ast$

\begin{itemize}[align = left, leftmargin=*]
	\item[\textbf{Key generation.}]	Randomly choose a secret key $x \in \ZZ_p^\ast$, and compute the public key $y = g^x$.
	
	\item[\textbf{Signature.}]To sign a message $m$, the signer chooses a random $k \in \ZZ_p^\ast$.
Compute $ r = g^k$. To compute $s$, the following equation must be satisfied: $g^{H(m)} = g^{xr} g^{ks}$.
So $s = \left( H(m) - xr \right) k^{-1} \quad (\text{mod } p-1)$

If $s=0$, it starts over again with a different $k$.

The pair $(r,s)$ is the digital signature for $m$.

	\item[\textbf{Verification.}]Check $g^{H(m)} = y^r r^s$

The use of $H( \cdot )$ prevents an existential forgery attack.

\end{itemize}

\subsubsection*{Boneh-Lynn-Shacham (BLS)}
\cite{BonehLS01}
Let $G_1,G_2$ be a bilinear group pair of prime order $p$. Let $g$ be a generator of $G_1$. Let $e:G_1 \times G_2 \rightarrow G_T$ be a non-degenerate bilinear pairing. Let $H: \{0,1\}^* \rightarrow G_1$ be a full-domain hash function.

\begin{itemize}[align = left, leftmargin=*]
	\item[\textbf{Key generation.}] Randomly choose a secret key $x \in \ZZ_p$. The public key is $y = g_2^x$.
	
	\item[\textbf{Signature.}] Given a private key $x \in \ZZ_p$, and a message $m \in \{0,1\}^*$, compute $h = H(m) \in G_1$ and $\sigma = h^x$. The signature is $\sigma \in G_1$.
	
	\item[\textbf{Verification.}] Given a public key $y$, a message $m \in \{0,1\}^*$ and a signature $\sigma \in G_1$, compute $h = H(m) \in G_1$ and verify that $e(\sigma,g_2) = e \left( h, y \right)$.
	
\end{itemize}

\subsubsection*{Schnorr}
\cite{Schnorr90}
Let $G$ be a cyclic group of prime order $p$. Let $g$ be a generator of $G$. Let $H : \{0,1\}^\ast \rightarrow \ZZ_p$ be a hash function.

\begin{itemize}[align = left, leftmargin=*]
	\item[\textbf{Key generation.}]	Randomly choose a secret key $x \in \ZZ_p$. The public key will be $y = g^x$.
	\item[\textbf{Signature.}] Randomly choose $z \in  \ZZ_p$, and compute $L := g^z$. \\
Compute $c := H(L \parallel m)$. \\
Compute $s := z + c \cdot x$ \\
The signature on $m$ is $\sigma = (c,s)$.
	\item[\textbf{Verification.}] Given a signature $\sigma$ and a public key $y$, computes $L^{\dagger} := g^{s} y^{-c}$ and then check that $c = H(L^{\dagger} \parallel m) $
\end{itemize}

\section{Signature Aggregation}
Explain how different signature schemes allow aggregation of n signatures on n messages from n signers.

\newpage
\section{Group Signatures}
Use of aggregation: group signatures. They are used to sign on behalf of the group, prove group membership.

The easiest way is to give everyone the secret key, so they can sign. But this would let any colluded user to share the secret key to other parties, which is not admissible.

Some group signatures need what is called a Dealer, which will deal with the keys.

Examples of Group signatures:

\section{Shamir Secret Sharing}
Shamir secret sharing described in \cite{Sham79}.

The scheme is based on polynomial interpolation.

Let $p$ be a large prime number. All operations are done in $\ZZ_p$

We want to share a secret $s$ into $n$ shares so that the secret can be recovered with any $k$ distinct shares.

Let $q(x) = a_0 + a_1 x + \cdots + a_{k-1} x^{k-1}$ be a random polynomial of degree $k-1$ in which $a_0 = s$.

Each participant $i$ in $\PP = \{ 1, ... , n \}$ is given a different random number $x_i \in \ZZ_p$ which identifies the participant.
Then, each participant $i$ is given the share $y_i = q(x_i)$.

To recover the secret, we only need $k$ different shares. Let $P \subset \PP$ be any subset of $k$ participants. Then

Let $\lambda_i^{P} = \prod_{P_j \in (P \setminus P_i)} \frac{-x_j}{x_i - x_j}$

$$
    q(x) = \sum_{i \in P} y_i \prod_{\substack{j \in P \\ i \neq j}} \frac{x-x_j}{x_i-x_j}
$$

Then, $s = q(0)$.

\section{Threshold signature scheme using Shamir}
\label{sec:shamir_sig}

This signature scheme will be based on the BLS scheme.

Let $G$ be a gap group of some prime order $p$. Let $g \in G$ be a generator of the group.

Let $\mathcal{P}= \{ 1, \dots , n \}$ be the set of participants in this scheme. Suppose every participant $P_i \in \mathcal{P}$ is given a distinct random number $x_i \in \ZZ_p$ as in the Shamir Secret Sharing Scheme.

Let $SK \in \ZZ_p$ be the secret key of the scheme. Let $q(x) = \sum_{i=0}^{t-1} a_i x^i$ be a random polynomial of degree $t-1$ but fixing $a_0 = SK$.

Each participant $P_i$ is given the share $s_i = q(x_i)$.

Then, for any set $P \subseteq \PP$ of $t$ participants $$SK = \sum_{i \in P} s_i \lambda_i^P$$

To sign a message $m$, each participant $P_i$ computes his partial signature $\sigma_i (m) = H(m)^{s_i}$ and broadcasts the pair $(x_i, \ \sigma_i (m))$.

Then, after a set $P$ of at least $t$ participants has broadcast their partial signatures for the message, a standard signature $\sigma$ can be computed:

$$ \sigma (m) = \prod_{P_i \in P} \sigma_i (m)^{\lambda_i^{P}}= H(m)^{\sum_{P_i \in P} \lambda_i^P s_i} = H(m)^{SK}$$

The signature is valid if $e(\sigma, g) = e(H(m), PK)$

\section{Anonymity}

In order to compare different schemes we need to clear up the definition of anonymity.

The word anonymity is derived from the Greek word \textit{anonymia}, meaning "without a name". In technical terms, the "name" of a participant would be something that uniquely identifies him, e.g. his public key. So, a scheme would be anonymous if the public key of the participant is not disclosed or cannot be obtained in any way at any moment.

It is not very intuitive how we can punt the concept of anonymity in a signature scheme. An anonymous signature does not make much sense. There is no use of an information signed by an anonymous person. What is useful is a signature from a known group of people, but cannot determine from which one nor distinguish two signatures of different participants of the same group on the same message.

- Absolute anonymity. No use. Same as no signature
- Anonymity inside a group. 



You could use ring signatures which proofs the knowledge of a 1-out-of-N secret key. This could be useful for a small amount of signatures. But it is not useful to provide general anonymity.

A signature scheme provides anonymity if:
\begin{itemize}
    \item Unlinkability: cannot decide whether two different signatures were signed by the same user.
    \item Untraceability: cannot get the public key of the signer from a valid signature.
\end{itemize}

\cite{BlSt97} 

An ideal secret sharing scheme is a scheme in which the size of the shares given to each participant is equal to the size of the secret.

In an anonymous secret sharing schemes the secret can be reconstructed without the knowledge of which participants hold which shares.