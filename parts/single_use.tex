\chapter{Single use: Anonymity}
\label{chap:single}
In this chapter we consider threshold signature schemes with non-traceability but with linkability.

As we commented previously, we can avoid the linkability of a threshold signature scheme by setting it up newly every time a signature is computed. We describe two examples.

\section{Anonymous Secret Sharing Scheme}
Consider a signature scheme with secret $sk$. Using the $(t,n)$-anonymous threshold secret sharing scheme described in \cite{BlSt97} we can share the secret $sk$ among the participants through a secure channel. When a set of $t$ participants want to sign a message, they can just recover the secret $sk$ and compute a signature with it.

After that, the signature scheme is newly setup by choosing a different secret $sk$ and sharing it again using an anonymous threshold secret sharing scheme.

In this solution, the dealer (the one who gives the shares to the participants) can know which participant holds which share because the participants have to authenticate themselves to avoid a single participant getting multiple shares. We can approach this by running a $1$-out-of-$N$ OT protocol randomly choosing a share for sufficiently large $N \gg n$ s.t. the probability of two participants choosing the same share is sufficiently low. In this way, the dealer does not know which participants holds which share, but the execution of an OT protocol for large $N$ is computationaly way too expensive.

A solution in this way for the e-voting case described in the introduction would be the following:
\begin{itemize}[align = left, leftmargin=*, label={--}]
\item Let $C_1, ..., C_k$ be the candidates to be voted, and let $\{P_1, ..., P_n\}$ be the set of voters. Each candidate needs at least $t$ votes to be validated. A voter can vote more than one candidate.
\item Let $p$ be a sufficiently large prime. For each $C_i$, the system chooses a random secret $sk_i \in_R \ZZ_p$ and a random polynomial $Q_i(x) = \sum_{j=0}^{t-1} a_{i,j} x^j$ of degree $t-1$ where $a_{i,0} = sk_i$.
\item Each candidate $P_j$ chooses $k$ random values $\alpha_{j,1}, ..., \alpha_{j,k} \in_R \ZZ_p$, then runs an oblivious polynomial evaluation protocol to learn the evaluations $Q_1(\alpha_{j,1}), ..., Q_k(\alpha_{j,k})$ from the system.
\item If a voter $P_j$ wants to vote a candidate $C_i$, just needs to share the pair $(\alpha_{j,i},Q_i(\alpha_{j,i})$.
\item At the end of the voting, if a candidate $C_i$ knows at least $t$ shares $(\alpha_{j,i},Q_i(\alpha_{j,i})$, he can interpolate $Q_i(x)$ to obtain the secret $sk_i = Q_i(0)$ and be validated.
\end{itemize}

\section{Anonymized Threshold BLS Signatures}
\label{sec:thr_bls}
Using the BLS threshold signature scheme described in \ref{sec:shamir_sig} we can avoid linkability setting up a new secret after a signature is computed.

After the signature scheme is set up with a new secret $sk$ and a new random polynomial $P(x)$, a participant $P_i$ chooses $\alpha_i \in_R \ZZ_p^\ast$ and learns $P(\alpha_i)$.

Recall that the threshold BLS signature scheme in section \ref{sec:shamir_sig} is not anonymous with respect to the dealer, who knows the $\alpha_i$ assigned to each participant $P_i$. A way to avoid this is to use an oblivious polynomial evaluation protocol (see section \ref{sec:ope}). In this way, $P_i$ obtains his secret share $P(\alpha_i)$ and the dealer obtains nothing on $\alpha_i$.

Given a set $\{P_1, ..., P_t\}$ of $t$ participants, there is a positive probability that two participants $P_i,P_j$ chose the same $\alpha_i = \alpha_j$, thus failing to perform a signature. If $\alpha_i$ are randomly chosen with uniform distribution among $\ZZ_p^\ast$, the probability that at least two participants hold the same share is $q = 1 - \prod_{k=1}^{t-1} (1-\frac{k}{p-1})$. Since $p \gg n$ and therefore $p \gg t$, we can approximate $q = \frac{t(t-1)}{2(p-1)} + o\left(\frac{t^2}{p}\right)$. If $p$ is a $1024$-bit prime (usual bit length for large primes in cryptography), and considering that there is no practical use for, lets say, $n > 2^{128}$ (the number of distinct IPv6 addresses) we can claim that $q < 2^{-769}$ in any practical case, which is absurdly small.