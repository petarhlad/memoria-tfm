\chapter{Anonymity in Threshold Signatures}
\label{chap:anon}
In order to compare different schemes we need to clear up the definition of anonymity.

The word anonymity is derived from the Greek word \textit{anonymia}, meaning "without a name". In technical terms, the "name" of a participant would be something that uniquely identifies him, e.g. his public key. So, a scheme would be anonymous if the public key of the participant is not disclosed or cannot be obtained in any way at any moment.

Many authors use the term \textit{anonymous threshold scheme} referring to a threshold scheme that does not require knowing which participant holds which share. The scheme proposed by \citeauthor{Boldyreva03} in \cite{Boldyreva03} does not require knowing which participant holds each share, but we cannot call it an anonymous scheme because there is only one participant $P_i$ holding the value $\alpha_i$ even though the proper identity of $P_i$ was not disclosed. Any party can check if a certain participant $P_i$ that holds the value $\alpha_i$ has participated in a signature $\sigma$ just by checking if $\alpha_i$ was used in the computation of $\sigma$. But still without knowing which participant holds it. We will call this property \textit{linkability}.

What we are looking for in this work, by saying \textit{anonymity}, is two properties:
\begin{itemize}
    \item Unlinkability: cannot decide whether two different signatures were signed by the same user.
    \item Untraceability: cannot get the public key of the signer from a valid signature.
\end{itemize}

A threshold signature scheme that has the linkability property is suitable for a "one-time anonymity". That is, once a signature has been computed, the whole scheme has to be set up again. We discuss solutions of this kind in chapter \ref{chap:single}.

A threshold signature scheme that is unlinkable is suitable for anonymity still after multiple uses. We discuss three solutions in chapter \ref{chap:mult}.
