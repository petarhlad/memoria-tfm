\chapter{Anonymity in Threshold Signatures}
Blabla Shamir is not anonymous

Single Use

Multiple use

In order to compare different schemes we need to clear up the definition of anonymity.

The word anonymity is derived from the Greek word \textit{anonymia}, meaning "without a name". In technical terms, the "name" of a participant would be something that uniquely identifies him, e.g. his public key. So, a scheme would be anonymous if the public key of the participant is not disclosed or cannot be obtained in any way at any moment.

It is not very intuitive how we can punt the concept of anonymity in a signature scheme. An anonymous signature does not make much sense. There is no use of an information signed by an anonymous person. What is useful is a signature from a known group of people, but cannot determine from which one nor distinguish two signatures of different participants of the same group on the same message.

- Absolute anonymity. No use. Same as no signature
- Anonymity inside a group. 



You could use ring signatures which proofs the knowledge of a 1-out-of-N secret key. This could be useful for a small amount of signatures. But it is not useful to provide general anonymity.

A signature scheme provides anonymity if:
\begin{itemize}
    \item Unlinkability: cannot decide whether two different signatures were signed by the same user.
    \item Untraceability: cannot get the public key of the signer from a valid signature.
\end{itemize}

\cite{BlSt97} 

An ideal secret sharing scheme is a scheme in which the size of the shares given to each participant is equal to the size of the secret.

In an anonymous secret sharing schemes the secret can be reconstructed without the knowledge of which participants hold which shares.