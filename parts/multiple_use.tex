\chapter{Multiple use: Anonymity and Non-traceability}
To achieve full anonymity we need some 
Multiple use 


\section{Constant size anonymous threshold signature}
\citeauthor*{DazaDSV09} proposed in \cite{DazaDSV09} an anonymous threshold signature scheme that sets a $(t,n)$-threshold signature scheme based on Shamir's secret sharing over a partition of the set $\PP$ of participants.

\subsection{Description}
\subsubsection*{Setup Algorithm}
\begin{itemize}[align = left, leftmargin=*, label={--}]
\item Let $\PP = \{P_1 , \dots , P_n \}$ be the set of participants. Consider $d$ distinct partitions of $\PP$ into $r$ parts.
$\PP^i = \{\PP^i_1, ... , \PP^i_r\}$ for $i \in \{1, ... , d\}$. This algorithm will set $d$ different threshold signature schemes, one for each partition of $\PP$.

\item Let $p > n$ be a sufficiently large prime. Let $sk_i \in_R \ZZ_p$ be the secret key of the $i$-th threshold signature scheme. Let $P_i(x)$ be a random polynomial over $\ZZ_p$ of degree $t-1$ with ${P_i(0) = \mbox{sk}_i}$ for $i \in \{1, ..., d\}$.

\item For $i \in \{1, ..., d\}$ let $\alpha^{(i)}_1, ... , \alpha^{(i)}_r \in_R \ZZ_p$ all distinct (for fixed $i$). Each participant $P_k \in \PP^i_j$ is given public key $pk^{(i)}_k = \alpha^{(i)}_j$ and secret key $sk^{(i)}_k = P_i(\alpha^{(i)}_j)$ for the $i$-th threshold signature scheme. 
\end{itemize}

\subsubsection*{Signing Algorithm}
\begin{itemize}[align = left, leftmargin=*, label={--}]
\item Let $\{P_{i_1}, ..., P_{i_t} \} \subseteq \PP$ a set of $t$ participants which will try to sign a message $m$.

\item A signature on $m$ over the $i$-th threshold signature scheme is attempted using the protocol described in section \ref{sec:shamir_sig} providing the message, the signature and the signature scheme over which is signed: $(m,\sigma, i)$.

\item If the signature fails (because $\alpha^{(i)}_{i_1}, ... , \alpha^{(i)}_{i_t}$ are not all distinct), a new signature on $m$ is attempted over a threshold signature scheme different from the previous tried.

\item Eventually, the signature will succeed over a certain signature scheme.
\end{itemize}

\subsubsection*{Verifying Algorithm}
\begin{itemize}[align = left, leftmargin=*, label={--}]
\item Let $(m, \sigma, i)$ be a signature on a message $m$. The verification is done using the verifying algorithm described in \ref{sec:shamir_sig}.
\item Let $e$ be a bilinear pairing. The signature is valid if and only if $e(\sigma,g) = e(H(m),pk_i)$.
\end{itemize}

\subsection{Analysis}
In this signature scheme it is not sure that any set of $t$ participants will be able to compute a signature on a given message. The probability of succeeding on signing a message depends on the parameters $t,n,r,d$ and how the partitions are made.

To see the relation between these parameters and the probability of success, we will describe a few examples.

Given $\{P_{i_1}, ... P_{i_t} \}$ a set of $t$ participants, they will succeed only if $\alpha^{(i)}_{i_1}, ..., \alpha^{(i)}_{i_t} $ are all distinct for a certain $i \in \{1, ...,d\}$. Hence, $t \leq r$.

\subsubsection*{Random partitions}
Suppose that for each participant $P_k \in \PP$ and for all $i \in \{1, ..., d\}$ the probability $Pr(P_k \in \PP^i_j) = \frac{1}{r}$.

\subsubsection*{Deterministic partitions}


To simplify the analysis, for each participant $P_k \in \PP$ we will consider the corresponding codeword in a (not necessary linear) code of length $d$ over an alphabet of size $r$ given by:
$$c_k := (j_1, ... , j_d) \text{ iff } P_k \in \PP^i_{j_i} \quad \forall i \in \{1, ..., d\}$$


\citeauthor{singleton} \cite{singleton}.

This sets up a $(t,r)$-threshold signature scheme.

When $t$ participants try to sign a message, there is a probability $p_{\text{fail}} = 1 - p_{\text{success}}$ that at least two participants belong to the same group. The probability that all $t$ participants belong to different groups is given by $p_{\text{success}} = \frac{\binom{r}{t} \left( \frac{n}{r} \right)^{t}}{\binom{n}{t}}$.

Looks like, for large values, $p_{\text{fail}} \sim \frac{t^2}{2r}$

In \cite{DazaDSV09} they extend this scheme in the following way: There are set $d$ different $(t,r)$-threshold signature scheme and every participant is given $d$ shares, one for each secret. The probability of not being able to sign a message is $\left(p_{\text{fail}} \right)^d$.

To be anonymous, we need $n \geq 2r$.

These schemes would be suitable for small values of $t$.

Some values:


\begin{center}
    \begin{tabular}{|c|r|c|r|r|c|}
        \hline
        $\frac{t^2}{2r}$ & \multicolumn{1}{c|}{$t$} & $\frac{n}{r}$ & \multicolumn{1}{c|}{$n$} & \multicolumn{1}{c|}{$r$} & $p_{fail}$ \\ \hline
        
        \multirow{4}{*}{$10^{-3}$} & $5$ & \multirow{4}{*}{$10^{3}$} & $12.5 \cdot 10^6$ & $12.5 \cdot 10^{3}$ & $0.80 \cdot 10^{-3}$ \\ \cline{2-2} \cline{4-6}
        & $10$ & & $50 \cdot 10^6$ & $50 \cdot 10^{3}$ & $0.90 \cdot 10^{-3}$ \\ \cline{2-2} \cline{4-6}
        & $50$ & & $1.25 \cdot 10^9$ & $1.25 \cdot 10^{6}$ & $0.98 \cdot 10^{-3}$ \\ \cline{2-2} \cline{4-6}
        & $100$ & & $5 \cdot 10^9$ & $5 \cdot 10^{6}$ & $0.99 \cdot 10^{-3}$ \\ \cline{2-2} \cline{4-6}
        
        \hline
    \end{tabular}
\end{center}

\cite{Deng04}

\section{Non-linear size anonymous threshold signature}
\cite{ChenNW11}
\subsection{Description}
\subsubsection*{Setup Algorithm}

Let $G_1$, $G_2$, $G_T$ be cyclic groups of sufficiently large prime order $q$. Two random generators $g_1 \in G_1$, $g_2 \in G_2$, and a bilinear pairing $\hat{t}: G_1 \times G_2 \rightarrow G_T$.

DDH problem in $G_1$, Gap-DL problem in $G_1$ and $G_2$ and the blind bilinear LRSW problem are hard.

Let $H_0 : \{0,1\}^\ast \rightarrow \ZZ_q$ and $H_1 : \{ 0 , 1 \}^\ast \rightarrow G_1$ be two hash functions.

For each issuer $i \in \mathcal{I}$ the following is performed.

Two integers are selected $x,y \in_R \ZZ_q$ and the issuer secret key \textbf{isk} is assigned to be $(x,y)$. Then the values $X = g_2^{x} \in G_2$ and $Y = g_2^{y} \in G_2$ are computed. The issuer public key \textbf{ipk} is assigned to be $(X,Y)$.

Finally the system public parameters $par$ are set to be $par = (G_1, G_2, G_T, \hat{t}, g_1, g_2, H_0, H_1, \text{ipk}_k)$ and are published.

\subsubsection*{Join protocol}

This is a protocol between a given signer $s \in S$  and an issuer $i \in \mathcal{I}$.


(Maybe could be a random $f \in G_1$)
The signer generates a secret value $f$ using its internal seed \textbf{TAAseed}, along with the value \textbf{K}\textsubscript{I} provided by $i$ and a count number \textbf{cnt}.

\begin{figure}[H]
$$
\begin{array}{ccc}
    \text{Signer}(\mathfrak{s}) &    & \text{Issuer}(\mathfrak{i}) \\
    \hline
    \\
    f \in_R \ZZ_q, \ F = g_1^f &    &    \\
    \text{str} \leftarrow X \parallel Y \parallel n_I &  \xleftarrow{\text{comm}_{\text{req}}} & \text{comm}_{\text{req}} \leftarrow n_I \\
        & \longrightarrow & \text{If } F = g_1^{f_i} \text{ for any } f_i \text{ on the roge list then \textbf{abort}} \\
        &        & r\in_R \ZZ_q; \ A = g_1^r; \ B = A^y \\
    \text{If } \hat{t}(A,Y) \neq \hat{t}(B,g_2) & \xleftarrow{\text{  cre  }} & C = (A^x \cdot F^{rxy}); \ cre \leftarrow(A,B,C) \\
    \text{or } \hat{t}(A \cdot B^f , X) \neq \hat{t}(C,g_2) &    & \\
    \text{then \textbf{abort}} &    & \\
    \hline
\end{array} 
$$
\caption{The Join Protocol}
\label{fig:join}
\end{figure}

\begin{figure}[H]
$$\begin{array}{l}
\text{Signer}(\mathfrak{s}) \\
\hline
\text{Input: } f \in \ZZ_q; \ n_T \leftarrow \{0,1\}^\ast; \ msg \\
a \in_R \ZZ_q; \ z \in_R \ZZ_q \\
J \leftarrow H_1(\text{msgt}); \ K = J^f; \ L = J^z \\
R = A^a; \ S = B^a; \ T = C^a; \ \tau = \hat{t}(S,X)^z \\
c \leftarrow H_0(R \parallel S \parallel T \parallel \tau \parallel J \parallel K \parallel L \parallel n_T \parallel \text{msgb}) \\
s \leftarrow z + c \cdot f \mod q \\
\sigma \leftarrow (R,S,T,J,K,c,s,n_T) \\
\text{Output: } \sigma \\
\hline
\end{array}
$$
\caption{The Sign Algorithm}
\label{fig:sign}
\end{figure}

\begin{figure}[H]
$$\begin{array}{l}
\text{Verifier}(\mathfrak{v}) \\
\hline
\text{Input: \textbf{ipk}}_k = (X,Y); \ \text{msg} = (\text{msgt}, \text{msgb}) \\
\sigma = (R,S,T,J,K,c,s,n_T) \\
\text{If } K = J^{f_i}, \text{ for any } f_i \text{in the set of rogue secret keys, or} \\
\qquad \hat{t}(R,Y) \neq \hat{t}(S,g_2), \text{ or} \\
\qquad J \neq H_1(\text{msgt}) \text{ return \textbf{reject}} \\
\rho_a^\dagger = \hat{t}(R,X); \rho_b^\dagger = \hat{t}(S,X); \rho_c^\dagger = \hat{t}(T,g_2) \\
\tau^\dagger = (\rho_b^\dagger)^s \cdot (\rho_c^\dagger / \rho_a^\dagger)^{-c} \\
L^\dagger = J^s \cdot K^{-c} \\
\text{If } c \neq H_0 (R \con S \con T \con \tau^\dagger \con J \con K \con L^\dagger \con n_T \con \text{msgb}) \text{ return \textbf{reject}} \\
\text{Otherwise return \textbf{accept}} \\
\hline
\end{array}
$$
\caption{The Verify Algorithm}
\label{fig:verify}
\end{figure}

In the Sign Algorithm (Fig. \ref{fig:sign}), the computation of $L$, $c$ and $s$ is for the non-interactive Zero-knowledge proof of knowledge of $f$, and $\tau$ is to proof the knowledge of the issuer's credentials. This is based on the \textit{Schnorr protocol} described in \cite{Schnorr90}

To see if two signatures with same message title were signed by the same signer, only have to check whether $J_0 = J_1$ and $K_0 = K_1$.

In the case of using this scheme in a polling system, \textbf{msgt} could be used to identify the poll and \textbf{msgb} for the poll choice.

\section{Anonymous interactive protocol}

We propose an interactive protocol based on the signature scheme described in section \ref{sec:shamir_sig}. Recall that this scheme does not have the \textit{unlinkability} property because it uses "pseudonyms" and they are shared to compute the signature. Thus, the idea of this new protocol is to hide these "pseudonyms".

As in the \textbf{¿previous?} scheme, a participant $P_i$ from a subset $P \subset \PP$ of $t$ participants will compute a partial signature $\sigma_i (m)$ on a message $m$. The partial signature will be $\sigma_i (m) = H(m)^{s_i \prod_{P_j \in (P \setminus P_i)} \frac{-\alpha_j}{\alpha_i - \alpha_j}}$.

The goal of this modification is to compute $a^\frac{-\alpha_j}{\alpha_i - \alpha_j}$ without sharing the values of $\alpha_i$ and $\alpha_j$, for $1 \neq a \in G$ and a given $P_j \in P$. \sout{The protocol needs $t^2$ interactions as the one described in figure \ref{fig:proposed_protocol} to be able to compute the whole signature.}

To compute $a^\frac{-\alpha_j}{\alpha_i - \alpha_j}$ we need participants $P_i,P_j$ and a third party $P_s$ that could be any other participant or a reliable party (like secure hardware).

\subsection{Description}
\subsubsection*{Interaction}
Let $a^{\frac{-\alpha_j}{\alpha_i - \alpha_j}} \leftarrow \mathcal{B}(a,P_i,P_j)$ the protocol that outputs $a^{\frac{-\alpha_j}{\alpha_i - \alpha_j}}$ given $1 \neq a \in G$, a first participant $P_i$ and a second participant $P_j$.

This protocol is split in five steps. The description follows:

\textbf{First step:} $P_i$ chooses $x_i, x_j, x_s \in \ZZ^*_p$ three random values and shares them with $P_j$ and $P_s$. These will be the new "pseudonyms".

$P_i$ and $P_j$ randomly choose polynomials $f_i, g_i, z_i$ and $f_j, g_j, z_j$, respectively, where: $f_i,f_j$ and $g_i,g_j$ are linear, $g_i(0) = \alpha_i$ and $g_j(0) = \alpha_j$, and $z_i,z_j$ are quadratic polynomials with $z_i(0) = z_j(0) = 0$.

\begin{figure}
        \begin{center}
        \begin{tikzpicture}
            
            \tikzset{vertex/.style = {draw=none,minimum size=1.5em}}
            \tikzset{calc/.style = {rectangle, draw}}
            \tikzset{edge/.style = {->,thick, > = latex'}}
        
            \node[vertex] (Pi) at (0,0) {$P_i$};
            \node[vertex] (Pj) at (4,0) {$P_j$};
            \node[vertex] (Ps) at (2,2.6) {$P_s$};
            \node[vertex] (BC) at (2,1) {BC};
            
            \node[calc, left = of Pi] (i) {
                \begin{tabular}{c}
                    $x_i,x_j,x_s \in_R \ZZ_p^{\ast}$ \\
                    $\gamma_{i,0}, \gamma_{i,1}, \gamma_{i,2} \in_R \ZZ_p$ \\
                    $g_i(x) \leftarrow \gamma_{i,1} \cdot x + \gamma_{i,0}$ \\
                    $f_i(x) \leftarrow \gamma_{i,2} \cdot x + \alpha_i$
                \end{tabular}
            };
            
            \node[calc, right = of Pj] (j) {
                \begin{tabular}{c}
                    $\gamma_{j,0}, \gamma_{j,1}, \gamma_{j,2} \in_R \ZZ_p$ \\
                    $g_j(x) \leftarrow \gamma_{j,1} \cdot x + \gamma_{j,0}$ \\
                    $f_j(x) \leftarrow \gamma_{j,2} \cdot x + \alpha_j$
                \end{tabular}
            };
            
            \draw[edge] (Pi) to node[sloped,midway,above] {$x_i,x_j,x_s$} (BC);
            
        
        \end{tikzpicture}
        \end{center}

\caption{Step 1}
\end{figure}

\textbf{Second step:} Let $h(x) = (g_i(x) + g_j(x))(f_i(x) - f_j(x)) + z_i(x) + z_j(x)$. Note that $h(0) = v \cdot (\alpha_i - \alpha_j)$ for $v := g_i(0) + g_j(0)$.

$P_i$ and $P_j$ share with the rest the evaluations of the random polynomials s.t. $P_i,P_j,P_s$ can compute the evaluations $h(x_i),h(x_j),h(x_s)$ respectively.

\begin{figure}
        \begin{center}
        \begin{tikzpicture}
            
            \tikzset{vertex/.style = {draw=none,minimum size=1.5em}}
            \tikzset{calc/.style = {rectangle, draw}}
            \tikzset{edge/.style = {->,thick, > = latex'}}
        
            \node[vertex] (Pi) at (0,0) {$P_i$};
            \node[vertex] (Pj) at (4,0) {$P_j$};
            \node[vertex] (Ps) at (2,2.6) {$P_s$};
            \node[vertex] (BC) at (2,1) {BC};
            
            \node[calc, left = of Pi] (i) { \footnotesize
                \begin{tabular}{c}
                    For $k \in \{i,j,s\}$ \\
                    $g_{ik} \leftarrow g_i(x_k)$ \\
                    $f_{ik} \leftarrow f_i(x_k)$ \\
                    $z_{ik} \leftarrow z_i(x_k)$
                \end{tabular}
            };
            
            \node[calc, right = of Pj] (j) { \footnotesize
                \begin{tabular}{c}
                    For $k \in \{i,j,s\}$ \\
                    $g_{jk} \leftarrow g_j(x_k)$ \\
                    $f_{jk} \leftarrow f_j(x_k)$ \\
                    $z_{jk} \leftarrow z_j(x_k)$
                \end{tabular}
            };
            
            \draw[edge] (Pi) to [bend left=10] node[sloped,midway,above] {$g_{ij},f_{ij},z_{ij}$} (Pj);
            \draw[edge] (Pj) to [bend left=10] node[sloped,midway,below] {$g_{ji},f_{ji},z_{ji}$} (Pi);
            \draw[edge] (Pi) to node[sloped,midway,above] {$g_{is},f_{is},z_{is}$} (Ps);
            \draw[edge] (Pj) to node[sloped,midway,above] {$g_{js},f_{js},z_{js}$} (Ps);
            
        
        \end{tikzpicture}
        \end{center}
\caption{Step 2}
\end{figure}

\textbf{Third step:} $P_i$,$P_j$,$P_s$ compute the evaluation of $h$ and share it with the rest. $P_i$ and $P_j$ interpolate the value $h(0)$.

\begin{figure}
        \begin{center}
        \begin{tikzpicture}[node distance=0.5cm]
            
            \tikzset{vertex/.style = {draw=none,minimum size=1.5em}}
            \tikzset{calc/.style = {rectangle, draw}}
            \tikzset{edge/.style = {->,thick, > = latex'}}
        
            \node[vertex] (Pi) at (0,0) {$P_i$};
            \node[vertex] (Pj) at (4,0) {$P_j$};
            \node[vertex] (Ps) at (2,2.6) {$P_s$};
            \node[vertex] (BC) at (2,1) {BC};
            
            \node[calc] (i) at (-2.5,-1){ \footnotesize
                \begin{tabular}{c}
                    $h_i \leftarrow (g_{ii} + g_{ji})(f_{ii} - f_{ji}) + z_{ii} + z_{ji}$ \\
                    $h \leftarrow \sum_{k \in \{i,j,s\}} h_k \prod_{\ell \neq k} \frac{- x_\ell}{x_k - x_\ell}$
                \end{tabular}
            };
            
            \node[calc] (j) at (6.5,-1){ \footnotesize
                \begin{tabular}{c}
                    $h_j \leftarrow (g_{ij} + g_{jj}) (f_{ij} - f_{jj}) + z_{ij} + z_{jj}$ \\
                    $h \leftarrow \sum_{k \in \{i,j,s\}} h_k \prod_{\ell \neq k} \frac{- x_\ell}{x_k - x_\ell}$
                \end{tabular}
            };
            
            \node[calc, above = of Ps] (s) { \footnotesize
                \begin{tabular}{c}
                    $h_s \leftarrow (g_{is} + g_{js}) (f_{is} - f_{js}) + z_{is} + z_{js}$
                \end{tabular}
            };
            
            \draw[edge] (Pi) to node[midway,below] {$h_i$} (BC);
            \draw[edge] (Pj) to node[midway,below] {$h_j$} (BC);
            \draw[edge] (Ps) to node[midway,right] {$h_s$} (BC);
            
        
        \end{tikzpicture}
        \end{center}
\caption{Step 3}
\end{figure}

\textbf{Fourth step:} $P_i$,$P_j$ compute $A_i=a^{\frac{1}{h(0)}(g_i(x_i)+g_j(x_i))}$, $A_j=a^{\frac{1}{h(0)}(g_i(x_j)+g_j(x_j))}$ respectively. $P_j$ shares $A_j$ with $P_i$.

$P_j$ can interpolate the exponents of $A_i$ and $A_j$ and compute $a^{\frac{g_i(0)+g_j(0)}{h(0)}} = a^{\frac{v}{v(\alpha_i - \alpha_j)}} = a^{\frac{1}{\alpha_i-\alpha_j}}$.

\begin{figure}
        \begin{center}
        \begin{tikzpicture}[node distance=0.5cm]
            
            \tikzset{vertex/.style = {draw=none,minimum size=1.5em}}
            \tikzset{calc/.style = {rectangle, draw}}
            \tikzset{edge/.style = {->,thick, > = latex'}}
        
            \node[vertex] (Pi) at (0,0) {$P_i$};
            \node[vertex] (Pj) at (4,0) {$P_j$};
            
            \node[calc, left = of Pi] (i) { \footnotesize
                \begin{tabular}{c}
                    $A_i \leftarrow a^{\frac{1}{h}(g_{ii} + g_{ji})}$ \\
                \end{tabular}
            };
            
            \node[calc, right = of Pj] (j) { \footnotesize
                \begin{tabular}{c}
                    $A_j \leftarrow a^{\frac{1}{h}(g_{ij} + g_{jj})}$ \\
                    $B \leftarrow A_i^{\frac{-x_j}{x_i-x_j}} A_j^{\frac{-x_i}{x_j-x_i}}$
                \end{tabular}
            };
            
            \draw[edge] (Pi) to node[sloped,midway,above] {$A_i$} (Pj);
                        
        
        \end{tikzpicture}
        \end{center}
\caption{Step 4}
\end{figure}

\textbf{Fifth step:} $P_j$ computes $B^{-\alpha_j} = a^{\frac{-\alpha_j}{\alpha_i - \alpha_j}}$ and shares it with $P_i$. 

\begin{figure}
        \begin{center}
        \begin{tikzpicture}[node distance=0.5cm]
            
            \tikzset{vertex/.style = {draw=none,minimum size=1.5em}}
            \tikzset{calc/.style = {rectangle, draw}}
            \tikzset{edge/.style = {->,thick, > = latex'}}
        
            \node[vertex] (Pi) at (0,0) {$P_i$};
            \node[vertex] (Pj) at (4,0) {$P_j$};
                        
            \node[calc, left = of Pi] (i) { \footnotesize
                \begin{tabular}{c}
                    Output: $B'$
                \end{tabular}
            };
            
            \node[calc, right = of Pj] (j) { \footnotesize
                \begin{tabular}{c}
                    $B' \leftarrow B^{-\alpha_j}$
                \end{tabular}
            };
            
            \draw[edge] (Pj) to node[sloped,midway,above] {$B'$} (Pi);
                        
        
        \end{tikzpicture}
        \end{center}
\caption{Step 5}
\end{figure}

\subsubsection*{Partial signature}
Let $P = \{P_i, P_{j_1}, \dots , P_{j_{t-1}} \}$.

For $P_i$ to compute the partial signature over a message $m$, computes $\sigma_i(m) = a_{t-1}$ where $a_k \leftarrow \mathcal{B}(a_{k-1},P_i,P_{j_k})$ for $k \in \{1, ... , t-1 \}$ and $a_0 = H(m)^{s_i}$

\subsubsection*{Signature}
The signature $\sigma(m)$ on a message $m$ from a group of $t$ participants $P = \{P_1, ... , P_t \}$ is
$$\sigma(m) = \prod_{i=1}^m \sigma_i(m)$$.

$$
\sigma_i (m)
= a_{t-1} {}^{\frac{- \alpha_{j_{t-1}}}{\alpha_{i} - \alpha_{j_{t-1}}}}
= a_k {}^{\frac{- \alpha_{j_{k}}}{\alpha_{i} - \alpha_{j_{k}}} \cdots \frac{- \alpha_{j_{t-1}}}{\alpha_{i} - \alpha_{j_{t-1}}}}
= a_1 {}^{\frac{- \alpha_{j_{1}}}{\alpha_{i} - \alpha_{j_{1}}} \cdots \frac{- \alpha_{j_{t-1}}}{\alpha_{i} - \alpha_{j_{t-1}}}}
= H(m)^{ s_i \frac{- \alpha_{j_{1}}}{\alpha_{i} - \alpha_{j_{1}}} \cdots \frac{- \alpha_{j_{t-1}}}{\alpha_{i} - \alpha_{j_{t-1}}}}
$$

\subsection{Analysis}
\subsubsection*{Correctness}
\textbf{Interaction:}
