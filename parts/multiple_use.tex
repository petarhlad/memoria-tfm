\chapter{Multiple use: Anonymity and Non-traceability}
To achieve full anonymity we need some 
Multiple use \cite{ChenNW11}
\cite{DazaDSV09}

Setup Algorithm

$G_1$, $G_2$, $G_T$ of sufficiently large prime order $q$. Two random generators $g_1 \in G_1$, $g_2 \in G_2$, and a bilinear pairing $\hat{t}: G_1 \times G_2 \rightarrow G_T$.

DDH problem in $G_1$, Gap-DL problem in $G_1$ and $G_2$ and the blind bilinear LRSW problem are hard.

Let $H_0 : \{0,1\}^\ast \rightarrow \ZZ_q$ and $H_1 : \{ 0 , 1 \}^\ast \rightarrow G_1$ be two hash functions.

For each issuer $i \in \mathcal{I}$ the following is perfomed.

Two integers are selected $x,y \in_R \ZZ_q$ and the isuer secret key \textbf{isk} is assigned to be $(x,y)$. Then the values $X = g_2^{x} \in G_2$ and $Y = g_2^{y} \in G_2$ are computed. The issuer public key \textbf{ipk} is assigned to be $(X,Y)$.

Finally the system public parameters $par$ are set to be $(G_1, G_2, G_T, \hat{t}, g_1, g_2, H_0, H_1, \text{ipk
}_k)$ and are published.

Join protocol

This is a protocol between a given signer $s \in S$  and an issuer $i \in \mathcal{I}$.


(Maybe could be a random $f \in G_1$)
The signer generates a secret value $f$ using its internal seed \textbf{TAAseed}, along with the value \textbf{K}\textsubscript{I} provided by $i$ and a count number \textbf{cnt}.

\begin{figure}[H]
$$
\begin{array}{ccc}
    \text{Signer}(\mathfrak{s}) &    & \text{Issuer}(\mathfrak{i}) \\
    \hline
    \\
    f \in_R \ZZ_q, \ F = g_1^f &    &    \\
    \text{str} \leftarrow X \parallel Y \parallel n_I &  \xleftarrow{\text{comm}_{\text{req}}} & \text{comm}_{\text{req}} \leftarrow n_I \\
        & \longrightarrow & \text{If } F = g_1^{f_i} \text{ for any } f_i \text{ on the roge list then \textbf{abort}} \\
        &        & r\in_R \ZZ_q; \ A = g_1^r; \ B = A^y \\
    \text{If } \hat{t}(A,Y) \neq \hat{t}(B,g_2) & \xleftarrow{\text{  cre  }} & C = (A^x \cdot F^{rxy}); \ cre \leftarrow(A,B,C) \\
    \text{or } \hat{t}(A \cdot B^f , X) \neq \hat{t}(C,g_2) &    & \\
    \text{then \textbf{abort}} &    & \\
    \hline
\end{array} 
$$
\caption{The Join Protocol}
\label{fig:join}
\end{figure}

\begin{figure}[H]
$$\begin{array}{l}
\text{Signer}(\mathfrak{s}) \\
\hline
\text{Input: } f \in \ZZ_q; \ n_T \leftarrow \{0,1\}^\ast; \ msg \\
a \in_R \ZZ_q; \ z \in_R \ZZ_q \\
J \leftarrow H_1(\text{msgt}); \ K = J^f; \ L = J^z \\
R = A^a; \ S = B^a; \ T = C^a; \ \tau = \hat{t}(S,X)^z \\
c \leftarrow H_0(R \parallel S \parallel T \parallel \tau \parallel J \parallel K \parallel L \parallel n_T \parallel \text{msgb}) \\
s \leftarrow z + c \cdot f \mod q \\
\sigma \leftarrow (R,S,T,J,K,c,s,n_T) \\
\text{Output: } \sigma \\
\hline
\end{array}
$$
\caption{The Sign Algorithm}
\label{fig:sign}
\end{figure}

\begin{figure}[H]
$$\begin{array}{l}
\text{Verifier}(\mathfrak{v}) \\
\hline
\text{Input: \textbf{ipk}}_k = (X,Y); \ \text{msg} = (\text{msgt}, \text{msgb}) \\
\sigma = (R,S,T,J,K,c,s,n_T) \\
\text{If } K = J^{f_i}, \text{ for any } f_i \text{in the set of rogue secret keys, or} \\
\qquad \hat{t}(R,Y) \neq \hat{t}(S,g_2), \text{ or} \\
\qquad J \neq H_1(\text{msgt}) \text{ return \textbf{reject}} \\
\rho_a^\dagger = \hat{t}(R,X); \rho_b^\dagger = \hat{t}(S,X); \rho_c^\dagger = \hat{t}(T,g_2) \\
\tau^\dagger = (\rho_b^\dagger)^s \cdot (\rho_c^\dagger / \rho_a^\dagger)^{-c} \\
L^\dagger = J^s \cdot K^{-c} \\
\text{If } c \neq H_0 (R \con S \con T \con \tau^\dagger \con J \con K \con L^\dagger \con n_T \con \text{msgb}) \text{ return \textbf{reject}} \\
\text{Otherwise return \textbf{accept}} \\
\hline
\end{array}
$$
\caption{The Verify Algorithm}
\label{fig:verify}
\end{figure}

In the Sign Algorithm (Fig. \ref{fig:sign}), the computation of $L$, $c$ and $s$ is for the non-interactive Zero-knowledge proof of knowledge of $f$, and $\tau$ is to proof the knowledge of the issuer's credentials.

To see if two signatures with same message title were signed by the same signer, only have to check whether $J_0 = J_1$ and $K_0 = K_1$.

In the case of using this scheme in a polling system, \textbf{msgt} could be used to identify the poll and \textbf{msgb} for the poll choice.