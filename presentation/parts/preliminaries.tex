\section{Preliminaries}

\begin{frame}{PKE scheme}
A public key encryption scheme $PKE = (KG, \E, \D)$ consists of three probabilistic and polynomial time algorithms:
\begin{itemize}
\item Key generation $KG$:
\begin{itemize}
\item Input: Security parameter
\item Output: Pair $(sk,pk)$ of secret and public keys.
\end{itemize}
\item Encription $\E$:
\begin{itemize}
\item Input: Plaintext $m$
\item Output: Ciphertext $c = \E_{pk}(m)$
\end{itemize}
\item Decryprtion $\D$:
\begin{itemize}
\item Input: Ciphertext $c$
\item Output: Plaintext $m = \D_{sk}(c)$
\end{itemize}
\end{itemize}

For any pair $(sk,pk)$ and any plaintext $m$, it must hold $$m = \D_{sk} \left( \E_{pk} (m) \right)$$

\end{frame}


\begin{frame}{Homomorphic PKE}

\begin{defn}[Homomorphic PKE]
Let $\M$ be the set of plaintexts s.t. it is closed under an operation $\bullet$. Let $\C$ be the set of ciphertexts s.t. it is closed under an operation $\circ$. A PKE scheme $(KG, \E, \D)$ has the homomorphic property if
$$
\D_{sk} \big( \E_{pk}(m_1) \circ \E_{pk}(m_2) \big) \ = \ m_1 \bullet m_2 \quad \forall m_1,m_2 \in \M.
$$
\end{defn}

\begin{rmk}
If we write $\M$ additively and $\C$ multiplicativelly, for $a \in \ZZ^+$ we have:
$$
\D_{sk} \left( \E_{pk} (m)^a \right) = a \cdot m
$$

\end{rmk}


\end{frame}

\subsection{Oblivious Polynomial Evaluation}
\begin{frame}{Oblivious Polynomial Evaluation}
Oblivious Polynomial Evaluation is a protocol involving a sender who knows a polynomial $P \in \FF[x]$ and a receiver who knows a value $\alpha \in \FF$. At the end of the protocol, the receiver learns $P(\alpha)$ and the sender learns nothing.


\begin{center}
\begin{tabular}{|c|c|c|}
\hline
             &    Sender & Receiver \\ \hline
    Input    &    $P \in \FF[x]$ & $\alpha \in \FF$ \\ \hline
    Output   &    -    &    $P(\alpha)$ \\ \hline

\end{tabular}
\end{center}

\end{frame}

\subsection{Bilinear Pairings}
\begin{frame}{Bilinear Pairings}

\begin{center}
\small
\begin{tabular}{|c|c|c|}
\hline
Problem name & Input & Output \\
\hline
Decisional DH (DDH) & $g,g^a,g^b,g^c \in G_1$ & TRUE iif $c=ab$ \\ \hline
Computational DH (CDH) & $g,g^a,g^b \in G_1$ &  $g^{ab}$  \\ \hline
Decisional Co-DH (co-DDH) & 
        $\begin{array}{r}
        h,h^b \in G_1 \\ 
        g_2, g_2^a \in G_2
        \end{array}$ & TRUE iif $a=b$ \\ \hline
Computational Co-DH (co-CDH) & 
        $\begin{array}{r}
        h \in G_1 \\
        g_2, g_2^a \in G_2
        \end{array}$ & $h^a$ \\ \hline

\end{tabular}
\end{center}
\end{frame}

\begin{frame}{Bilinear Pairings}
\begin{defn}[Bilinear map]
Let $G_T$ be an additional group s.t. $ |G_1| = |G_2| = |G_T| $. 

A \textbf{bilinear map} is a map $e: G_1 \times G_2 \rightarrow G_T$ s.t.:
\begin{itemize}
\item Is bilinear: $\forall u \in G_1$, $\forall v \in G_2$, $ \forall a,b \in \ZZ$, $$e(u^a, v^b) = e(u,v)^{ab} $$
\item Is non-degenerate: $e(g_1, g_2) \neq 1$.
\end{itemize}
\end{defn}

\end{frame}
\subsection{Secret Sharing}

\subsection{Digital Signatures}

\subsection{Group Signatures}

\subsection{Threshold Digital Signatures}