\section{Preliminaries}

\begin{frame}{PKE scheme}
A public key encryption scheme $PKE = (KG, \E, \D)$ consists of three probabilistic and polynomial time algorithms:
\begin{itemize}
\item Key generation $KG$:
\begin{itemize}
\item Input: Security parameter
\item Output: Pair $(sk,pk)$ of secret and public keys.
\end{itemize}
\item Encription $\E$:
\begin{itemize}
\item Input: Plaintext $m$
\item Output: Ciphertext $c = \E_{pk}(m)$
\end{itemize}
\item Decryprtion $\D$:
\begin{itemize}
\item Input: Ciphertext $c$
\item Output: Plaintext $m = \D_{sk}(c)$
\end{itemize}
\end{itemize}

For any pair $(sk,pk)$ and any plaintext $m$, it must hold $$m = \D_{sk} \left( \E_{pk} (m) \right)$$

\end{frame}


\begin{frame}{Homomorphic PKE}

\begin{defn}[Homomorphic PKE]
Let $\M$ be the set of plaintexts s.t. it is closed under an operation $\bullet$. Let $\C$ be the set of ciphertexts s.t. it is closed under an operation $\circ$. A PKE scheme $(KG, \E, \D)$ has the homomorphic property if
$$
\D_{sk} \big( \E_{pk}(m_1) \circ \E_{pk}(m_2) \big) \ = \ m_1 \bullet m_2 \quad \forall m_1,m_2 \in \M.
$$
\end{defn}

\begin{rmk}
If we write $\M$ additively and $\C$ multiplicativelly, for $a \in \ZZ^+$ we have:
$$
\D_{sk} \left( \E_{pk} (m)^a \right) = a \cdot m
$$

\end{rmk}


\end{frame}

\subsection{Oblivious Polynomial Evaluation}
\begin{frame}{Oblivious Polynomial Evaluation}
Oblivious Polynomial Evaluation is a protocol involving a sender who knows a polynomial $P \in \FF[x]$ and a receiver who knows a value $\alpha \in \FF$. At the end of the protocol, the receiver learns $P(\alpha)$ and the sender learns nothing.


\begin{center}
\begin{tabular}{|c|c|c|}
\hline
             &    Sender & Receiver \\ \hline
    Input    &    $P \in \FF[x]$ & $\alpha \in \FF$ \\ \hline
    Output   &    -    &    $P(\alpha)$ \\ \hline

\end{tabular}
\end{center}

\end{frame}

\subsection{Bilinear Pairings}
\begin{frame}{Bilinear Pairings}
Let $G_1$ and $G_2$ be two cyclic groups of prime order $q$. We write them multiplicativelly.
\begin{center}
\small
\begin{tabular}{|c|c|c|}
\hline
Problem name & Input & Output \\
\hline
Decisional DH (DDH) & $g,g^a,g^b,g^c \in G_1$ & TRUE iif $c=ab$ \\ \hline
Computational DH (CDH) & $g,g^a,g^b \in G_1$ &  $g^{ab}$  \\ \hline
Decisional Co-DH (co-DDH) & 
        $\begin{array}{r}
        h,h^b \in G_1 \\ 
        g_2, g_2^a \in G_2
        \end{array}$ & TRUE iif $a=b$ \\ \hline
Computational Co-DH (co-CDH) & 
        $\begin{array}{r}
        h \in G_1 \\
        g_2, g_2^a \in G_2
        \end{array}$ & $h^a$ \\ \hline

\end{tabular}
\end{center}
\end{frame}

\begin{frame}{Bilinear Pairings}
\begin{defn}[Bilinear map]
Let $G_T$ be an additional group s.t. $ |G_1| = |G_2| = |G_T| $. 

A \textbf{bilinear map} is a map $e: G_1 \times G_2 \rightarrow G_T$ s.t.:
\begin{itemize}
\item Is bilinear: $\forall u \in G_1$, $\forall v \in G_2$, $ \forall a,b \in \ZZ$, $$e(u^a, v^b) = e(u,v)^{ab} $$
\item Is non-degenerate: $e(g_1, g_2) \neq 1$.
\end{itemize}
\end{defn}
\begin{defn}[Gap problem]
A Gap co-Diffie-Hellman (co-GDH) group pair $(G_1,G_2)$ is s.t. co-DDH is easy but co-CDH is hard. When there is an efficient isomorphism $G_1 \cong G_2$ we say $G_1$ is a Gap group (GDH).

\end{defn}
\end{frame}



\subsection{Secret Sharing}
\begin{frame}{Secret Sharing}
$\PP := \{ P_1, ..., P_n\}$ set of participants.
\end{frame}

\begin{frame}{Shamir Secret Sharing}
Shamir (1979). Goal: Share a secret $s \in \ZZ_p$

\begin{itemize}
\item $s \in_R \ZZ_p$ the secret to be shared among $\PP$.
\item Set $a_0 = s$ and choose $a_1, ... , a_{t-1} \in_R \ZZ_p$ with $a_{t-1} \neq 0$ \\
        Set $P(x) = \sum_{i=0}^{t-1} a_i x^i$ polynomial of degree $t-1$
\item Choose $\alpha_1, ..., \alpha_n \in_R \ZZ_p^\ast$ all distinct.
\item Each participant $P_i \in \PP$ is given the share $(\alpha_i, y_i := P(\alpha_i)$
\item The secret can be recovered with at least $t$ shares with polynomial interpolation:
$$ s = P(0) \leftarrow \sum_{j=1}^t y_{i_j} \prod_{\substack{k \in [t] \setminus \{j\}}} \frac{-\alpha_{i_k}}{\alpha_{i_j}-\alpha_{i_k}}$$
\end{itemize}


\end{frame}

\begin{frame}{Anonymous Secret Sharing}
\begin{defn}
An anonymous secret sharing scheme is a secret sharing scheme in which the secret can be reconstructed without the knowledge of which participants hold which shares.
\end{defn}

\end{frame}


\subsection{Digital Signatures}
\begin{frame}{Digital Signatures}
A Digital Signature Scheme consists of 3 algorithms:
\begin{itemize}
\item Key Generation:
	\begin{itemize}
	    \item Input: Security parameter.
	    \item Output: Pair $(sk,pk)$ of secret and public keys.
	\end{itemize}
\item Sign:
	\begin{itemize}
	    \item Input: Message $m$, secret key $sk$.
	    \item Output: Signature $\sigma$ on the message $m$.
	\end{itemize}
\item Verify:
	\begin{itemize}
	    \item Input: Message $m$, signature $\sigma$ on $m$, public key $pk$.
	    \item Output: TRUE if the signature is valid. Otherwise FALSE.
	\end{itemize}
\end{itemize}
\end{frame}

\begin{frame}{BLS Signature Scheme}
Boneh, Lynn and Shacham (2001)

Let:
\begin{itemize}
\item[--] $(G_1,G_2)$ a bilinear group pair of prime order $p$
\item[--] $g$ a generator of $G_1$
\item[--] $e: G_1 \times G_2 \rightarrow G_T$ a bilinear pairing.
\item[--] $H: \{0,1\}^\ast \rightarrow G_1$ a full-domain hash function.
\end{itemize}
Key generation: Choose secret key $x \in_R \ZZ_p$. Set public key $y:=g_2^x$.

Sign: Given message $m \in \{0,1\}^\ast$, compute $h:= H(m) \in G_1$ and then compute the signature $\sigma:= h^x \in G_1$

Verify: Given public key $y$, message $m$ and signature $\sigma$, compute $h = H(m)$ and verify that $e(\sigma,g_2) = e(h,y)$.
\end{frame}

\subsection{Group Signatures}
\begin{frame}{Group Signatures}
Allow a member of the group to anonymously sign a message on behalf of the group
Properties:
\begin{itemize}
\item Unforgeability
\item Anonymity
\end{itemize}

Optional properties:
\begin{itemize}
\item Unlinkability
\item Traceability
\end{itemize}

\end{frame}

\subsection{Threshold Digital Signatures}
\begin{frame}{Threshold Digital Signatures}
\begin{defn}
A $(t,n)$-threshold signature scheme is a signature scheme in which any set of $t$ participants of the group is able to compute a signature on behalf of the group, and any subset of less than $t$ participants is unable to compute a valid signature.
\end{defn}

\end{frame}

\begin{frame}{Example of Threshold Signature}
Boldyreva (2003)

Setup Algorithm:
\begin{itemize}
\item $\PP=\{P_i\}$ set of $n$ participants.
\item $G$ a Gap group of large prime order $p > n$, and $g \in G$ a generator of the group.
\item Choose $sk \in_R \ZZ_p$ the secret key. Set $pk = g^{sk}$ the public key.
\item Set $a_0 = sk$ and choose $a_1, ..., a_{t-1} \in_R \ZZ_p$ with $a_{t-1} \neq 0$. \\
        Set $P(x) = \sum_{i=0}^{t-1} a_i x^i$.
\item Choose $\alpha_1, ..., \alpha_n \in_R \ZZ_p$ all distinct.
\item Each participant $P_i \in \PP$ is given public key $pk_i = \alpha_i$ and secret key $sk_i = P(\alpha_i)$
\end{itemize}

\end{frame}

\begin{frame}
Sign Algorithm:
\begin{itemize}
\item $P = \{P_{i_1}, ... P_{i_t} \}$ set of $t$ participants to sign message $m$.
\item Each $P_{i_j}$ computes partial signature $\sigma_{i_j}(m) = H(m)^{sk_{i_j}}$
\item Each $P_{i_j}$ broadcasts the pair $(pk_{i_j}, \sigma_{i_j}(m))$
\item The signature $\sigma$ on $m$ is computed:
$$ \sigma(m) = \prod_{P_i \in P} \sigma_i (m)^{\lambda_i^{P}}= H(m)^{\sum_{P_i \in P} \lambda_i^P sk_i} = H(m)^{sk}$$
where $\lambda_{i_j}^P := \prod_{k \in [t] \setminus \{j\}} \frac{-\alpha_{i_k}}{\alpha_{i_j} - \alpha_{i_k}}$.
\end{itemize}
\end{frame}

\begin{frame}
Verify Algorithm:
\begin{itemize}
\item $e: G \times G \rightarrow G_t$ bilinear pairing.
\item $\sigma$ signature on a message $m$.
\item $\sigma$ is valid $\Leftrightarrow e(\sigma,g) = e(H(m),pk)$ 
\end{itemize}
\end{frame}