\section{Multiple Use}
\begin{frame}{Multiple Use: Anonymity with Non-Linkability}
We describe three solutions:
\begin{itemize}
\item Constant Size Signature Scheme
\item Linkable Group Signature Scheme
\item Anonymous Interactive Protocol
\end{itemize}

\end{frame}

\begin{frame}{Constant Size Anonymous Threshold Signature}
Daza et al. (2009)

Setup Algorithm:
\begin{itemize}
\item Consider $d$ distinct partitions of the set of participants $\PP$ into $r$ parts: $\PP^i = \{\PP_1^i, ... , \PP_r^i\}$.
\item For each partition $\PP^i, i \in [d]$ set up a $(t,r)$-Threshold BLS Signature scheme, and give same key pairs to all participants in the same $\PP_j^i$
\end{itemize}

\end{frame}

\begin{frame}
Sign Algorithm
\begin{itemize}
\item $\{ P_{i_1}, ... , P_{i_t}\}$ set of $t$ participants to sign a message $m$.
\item Signature on $m$ over the i-th signature scheme is attempted. If succeeds, outputs $(m,\sigma,i)$.
\item If signature fails (at least two participants have same secret key), a new signature over a distinct signature scheme is attempted.
\item Eventually, the signature will succeed.
\end{itemize}

\end{frame}

\begin{frame}
Verify Algorithm:
\begin{itemize}
\item Signature $(m,\sigma,i)$.
\item Signature valid $\Leftrightarrow e(\sigma,g) = e(H(m),pk_i)$
\end{itemize}
\end{frame}

\begin{frame}{Linkable Group Signature Scheme}
Chen, Ng and Wang (2011)

Setup Algorithm:
\begin{itemize}
\item Participant generates a pair $(sk,pk)$ of secret and public keys.
\item The issuer gives the participant the credential $(A,B,C)$ that authenticates the participant's public key as member of the group.
\end{itemize}

\end{frame}