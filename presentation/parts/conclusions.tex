\section{Conclusions}
\begin{frame}{Conclusions}
Constant Size Anonymous Threshold Signature:
\begin{itemize}
\item Compact signature.
\item Unlinkability determined by the amount of participants in each part of the partitions.
\item Not always a group of $t$ participants can compute a signature, but we can control the probability of not succeeding.
\end{itemize}

Linkable Group Signature Scheme:
\begin{itemize}
\item Any set of $t$ participants can compute a signature.
\item Threshold is achieved by collecting $t$ unlinked signatures on the same message.
\item Length of the signature grows linearly with $t$.
\item Verification complexity is quadratic on $t$.
\end{itemize}


\end{frame}

\begin{frame}{Conclusions}
Anonymous Interactive Protocol:
\begin{itemize}
\item Unlinkable and untraceable whenever an adversary can corrupt at most one participant.
\item Compact signature.
\item Requires big amount of interactions: quadratic in $t$.
\item Constant signature verification time (independent of $t$).
\end{itemize}

Main problem of finding compact, non-interactive, unlinkable anonymous threshold siganture scheme remains open.
\end{frame}